%% Generated by Sphinx.
\def\sphinxdocclass{report}
\documentclass[letterpaper,10pt,english]{sphinxmanual}
\ifdefined\pdfpxdimen
   \let\sphinxpxdimen\pdfpxdimen\else\newdimen\sphinxpxdimen
\fi \sphinxpxdimen=.75bp\relax

\usepackage[utf8]{inputenc}
\ifdefined\DeclareUnicodeCharacter
 \ifdefined\DeclareUnicodeCharacterAsOptional
  \DeclareUnicodeCharacter{"00A0}{\nobreakspace}
  \DeclareUnicodeCharacter{"2500}{\sphinxunichar{2500}}
  \DeclareUnicodeCharacter{"2502}{\sphinxunichar{2502}}
  \DeclareUnicodeCharacter{"2514}{\sphinxunichar{2514}}
  \DeclareUnicodeCharacter{"251C}{\sphinxunichar{251C}}
  \DeclareUnicodeCharacter{"2572}{\textbackslash}
 \else
  \DeclareUnicodeCharacter{00A0}{\nobreakspace}
  \DeclareUnicodeCharacter{2500}{\sphinxunichar{2500}}
  \DeclareUnicodeCharacter{2502}{\sphinxunichar{2502}}
  \DeclareUnicodeCharacter{2514}{\sphinxunichar{2514}}
  \DeclareUnicodeCharacter{251C}{\sphinxunichar{251C}}
  \DeclareUnicodeCharacter{2572}{\textbackslash}
 \fi
\fi
\usepackage{cmap}
\usepackage[T1]{fontenc}
\usepackage{amsmath,amssymb,amstext}
\usepackage{babel}
\usepackage{times}
\usepackage[Bjarne]{fncychap}
\usepackage[dontkeepoldnames]{sphinx}

\usepackage{geometry}

% Include hyperref last.
\usepackage{hyperref}
% Fix anchor placement for figures with captions.
\usepackage{hypcap}% it must be loaded after hyperref.
% Set up styles of URL: it should be placed after hyperref.
\urlstyle{same}
\addto\captionsenglish{\renewcommand{\contentsname}{Contents:}}

\addto\captionsenglish{\renewcommand{\figurename}{Fig.}}
\addto\captionsenglish{\renewcommand{\tablename}{Table}}
\addto\captionsenglish{\renewcommand{\literalblockname}{Listing}}

\addto\captionsenglish{\renewcommand{\literalblockcontinuedname}{continued from previous page}}
\addto\captionsenglish{\renewcommand{\literalblockcontinuesname}{continues on next page}}

\addto\extrasenglish{\def\pageautorefname{page}}

\setcounter{tocdepth}{2}



\title{ASKE TA2 KApEESH Documentation}
\date{Mar 28, 2019}
\release{0.1}
\author{Natarajan Chennimalai Kumar}
\newcommand{\sphinxlogo}{\vbox{}}
\renewcommand{\releasename}{Release}
\makeindex

\begin{document}

\maketitle
\sphinxtableofcontents
\phantomsection\label{\detokenize{index::doc}}


DARPA’s Automating Scientific Knowledge Extraction (ASKE) topic area 2 focusing on machine assisted inference. Here we describe in detail
the execution module based on GE’s Dynamic Bayesian Network (DBN) framework.


\chapter{Module: DBN}
\label{\detokenize{index:module-dbn}}\label{\detokenize{index:welcome-to-aske-ta2-kapeesh-s-documentation}}

\bigskip\hrule\bigskip

\phantomsection\label{\detokenize{index:module-dbnrisk_wrapper_aske}}\index{dbnrisk\_wrapper\_aske (module)}
Wrapper module that integrates GE’s DBN framework as the execution engine for KaPEESH as part of DARPA ASKE Topic Area 2.
We utilize Python’s flask framework to design REST services that can be used through an API.

Written by Natarajan Chennimalai Kumar

General Electric Global Research Center

Mar 27, 2019
\index{RunDBNExecute() (in module dbnrisk\_wrapper\_aske)}

\begin{fulllineitems}
\phantomsection\label{\detokenize{index:dbnrisk_wrapper_aske.RunDBNExecute}}\pysiglinewithargsret{\sphinxbfcode{RunDBNExecute}}{\emph{aske\_input}, \emph{log\_instance}}{}
Primary DBN execution function. This function sets up the data structures to be used for the DBN execution
and performs some level of postprocessing.
\begin{quote}\begin{description}
\item[{Parameters}] \leavevmode\begin{itemize}
\item {} 
\sphinxstyleliteralstrong{aske\_input} \textendash{} Python dict object containing the DBN execution set up information

\item {} 
\sphinxstyleliteralstrong{log\_instance} \textendash{} A logging instance to keep track of the current status

\end{itemize}

\item[{Returns}] \leavevmode
model\_details: Python dict object containing the details and outputs from the DBN execution
pfdbn: DBN object contains the whole network, prior, posterior samples etc

\end{description}\end{quote}

\end{fulllineitems}

\index{StreamToLogger (class in dbnrisk\_wrapper\_aske)}

\begin{fulllineitems}
\phantomsection\label{\detokenize{index:dbnrisk_wrapper_aske.StreamToLogger}}\pysiglinewithargsret{\sphinxbfcode{class }\sphinxbfcode{StreamToLogger}}{\emph{logger}, \emph{log\_level=20}}{}
Fake file-like stream object that redirects writes to a logger instance.

\end{fulllineitems}

\index{as\_float() (in module dbnrisk\_wrapper\_aske)}

\begin{fulllineitems}
\phantomsection\label{\detokenize{index:dbnrisk_wrapper_aske.as_float}}\pysiglinewithargsret{\sphinxbfcode{as\_float}}{\emph{obj}}{}
Checks each dict passed to this function if it contains the key “value”
Args:
\begin{quote}

obj (dict): The object to decode
\end{quote}
\begin{description}
\item[{Returns:}] \leavevmode
dict: The new dictionary with changes if necessary

\end{description}

\end{fulllineitems}

\index{postProcessHypothesis() (in module dbnrisk\_wrapper\_aske)}

\begin{fulllineitems}
\phantomsection\label{\detokenize{index:dbnrisk_wrapper_aske.postProcessHypothesis}}\pysiglinewithargsret{\sphinxbfcode{postProcessHypothesis}}{\emph{model\_details}, \emph{aske\_input}}{}
This function takes the data after the DBN execution to analyze the hypothesis that the model can predict data

Arguments:
:param model\_details: Python dict object containing current update of details
:param aske\_input: original ubl input dict object
:return: model\_details: updated model details and input json with outputs from the hypothesis testing including error in prediction of the output nodes and aggregated error measures for model comparison

\end{fulllineitems}

\index{process() (in module dbnrisk\_wrapper\_aske)}

\begin{fulllineitems}
\phantomsection\label{\detokenize{index:dbnrisk_wrapper_aske.process}}\pysiglinewithargsret{\sphinxbfcode{process}}{}{}
This function runs the DBN execution as a flask REST service that will called by the KApEESH execution manager.
It sets up logging service, takes the POST request’s json as the input to the DBN execution.
\begin{quote}\begin{description}
\item[{Returns}] \leavevmode
model\_details as the response json

\end{description}\end{quote}

\end{fulllineitems}

\index{runDBN() (in module dbnrisk\_wrapper\_aske)}

\begin{fulllineitems}
\phantomsection\label{\detokenize{index:dbnrisk_wrapper_aske.runDBN}}\pysiglinewithargsret{\sphinxbfcode{runDBN}}{\emph{aske\_input}, \emph{log\_instance}}{}
Initializes logging and runs the DBN execution. In case of hypothesis generation usecase,
sets up the hypothesis and postprocessing.
\begin{quote}\begin{description}
\item[{Parameters}] \leavevmode\begin{itemize}
\item {} 
\sphinxstyleliteralstrong{aske\_input} \textendash{} Python dict object containing the DBN execution set up information

\item {} 
\sphinxstyleliteralstrong{log\_instance} \textendash{} A logging instance to keep track of the current status

\end{itemize}

\item[{Returns}] \leavevmode
model\_details: Python dict object containing the details and outputs from the DBN execution
pfdbn: DBN object contains the whole network, prior, posterior samples etc

\end{description}\end{quote}

\end{fulllineitems}



\chapter{Indices and tables}
\label{\detokenize{index:indices-and-tables}}\begin{itemize}
\item {} 
\DUrole{xref,std,std-ref}{genindex}

\item {} 
\DUrole{xref,std,std-ref}{modindex}

\item {} 
\DUrole{xref,std,std-ref}{search}

\end{itemize}


\renewcommand{\indexname}{Python Module Index}
\begin{sphinxtheindex}
\def\bigletter#1{{\Large\sffamily#1}\nopagebreak\vspace{1mm}}
\bigletter{d}
\item {\sphinxstyleindexentry{dbnrisk\_wrapper\_aske}}\sphinxstyleindexpageref{index:\detokenize{module-dbnrisk_wrapper_aske}}
\end{sphinxtheindex}

\renewcommand{\indexname}{Index}
\printindex
\end{document}